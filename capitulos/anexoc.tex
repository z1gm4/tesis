\chapter{Anexo C}
\lstset{language=C, 
showstringspaces=false,
backgroundcolor=\color{white},
breakatwhitespace=false,         % sets if automatic breaks should only happen at whitespace
  breaklines=true,                 % sets automatic line breaking
  captionpos=b,                    % sets the caption-position to bottom
  commentstyle=\color{mygreen},    % comment style
  deletekeywords={...},            % if you want to delete keywords from the given language
  escapeinside={\%*}{*)},          % if you want to add LaTeX within your code
  extendedchars=true,              % lets you use non-ASCII characters; for 8-bits encodings only, does not work with UTF-8
  frame=single,	                   % adds a frame around the code
  keepspaces=true,                 % keeps spaces in text, useful for keeping indentation of code (possibly needs columns=flexible)
  keywordstyle=\color{blue},       % keyword style
  language=Octave,                 % the language of the code
  morekeywords={*,...},           % if you want to add more keywords to the set
  numbers=left,                    % where to put the line-numbers; possible values are (none, left, right)
  numbersep=5pt,                   % how far the line-numbers are from the code
  numberstyle=\tiny\color{mygray}, % the style that is used for the line-numbers
  rulecolor=\color{black},         % if not set, the frame-color may be changed on line-breaks within not-black text (e.g. comments (green here))
  showspaces=false,                % show spaces everywhere adding particular underscores; it overrides 'showstringspaces'
  showstringspaces=false,          % underline spaces within strings only
  showtabs=false,                  % show tabs within strings adding particular underscores
  stepnumber=2,                    % the step between two line-numbers. If it's 1, each line will be numbered
  stringstyle=\color{mymauve},     % string literal style
  tabsize=2,	                   % sets default tabsize to 2 spaces
  title=\lstname  }
\begin{lstlisting}[frame=single,caption=Extracto de código del módulo de transición LoRa/IPv6	]  % Start your code-block
/* main loop */
    while (true) {
        /* fetch packets */
        if(tun_fd != -1){
         nread = read(tun_fd,buffer,sizeof(buffer));   
            RoHC_package = SCHC_TX(IPv6ToMesh(buffer,nread,lowpanh),(nread - 40) +27);
            //RoHC_package = IPv6ToMesh(buffer,nread,lowpanh);
            if(RoHC_package != NULL){
                printf("%s\n","manda mensaje");
                txpkt.size = (nread - 40) + 3;
                //txpkt.size = (nread - 40) + 27;
                for(j = 0; j < txpkt.size; j++ ){
                    txpkt.payload[j] = RoHC_package[j];
                    printf("%i-", RoHC_package[j]);}
                printf("\n");
                i = lgw_send(txpkt);

                wait_ms(delay);
        //wait_ms(1);
                nb_pkt = lgw_receive(ARRAY_SIZE(rxpkt), rxpkt);
                /* log packets */
                if(nb_pkt > 0){
                    printf("%s\n","LLega mensaje");
                    p = &rxpkt[0];
                    for(i = 0; i < p->size; i++){
                        printf("%i-", p->payload[i]);}
                    //Inicio de socket para enviar a bdd datos de LoRa//
                    printf("\n");
//Inicio de socket para enviar a bdd datos de LoRa//
MYSQL *con = mysql_init(NULL);
if (con == NULL){
	fprintf(stderr, "%s\n", mysql_error(con));
    exit(1);}
if (mysql_real_connect(con,"fe80::8dfe:9b57:3ea6:dbed","tesis", "tesis","testdb", 0, NULL, 0) == NULL){
	finish_with_error(con);} 
struct ifaddrs *ifaddr, *ifa;
int family, s;
char host[NI_MAXHOST];
if (getifaddrs(&ifaddr) == -1){
	perror("getifaddrs");
	exit(EXIT_FAILURE);}
for (ifa = ifaddr; ifa != NULL; ifa = ifa->ifa_next){
	if (ifa->ifa_addr == NULL)
		continue;  
	s=getnameinfo(ifa->ifa_addr,
	sizeof(struct sockaddr_in),host, NI_MAXHOST,
	 NULL, 0, NI_NUMERICHOST);
	if((strcmp(ifa->ifa_name,"eth0")==0)&&
	(ifa->ifa_addr->sa_family==AF_INET)){
    	 if (s != 0){
       	 	printf("getnameinfo() failed: 
       	 	%s\n", gai_strerror(s));
        	exit(EXIT_FAILURE);}
         printf("\tInterface:<%s>\n",ifa->ifa_name);
         printf("\t  Address:<%s>\n", host);}}
freeifaddrs(ifaddr);
	/*Insercion de ip de GW que envia datos,
	 y payload obtenido desde el nodo*/
	if (mysql_query(con,
	"INSERT INTO tesis VALUES(host, p->paylod)")){
		finish_with_error(con);
	}
    /*Impresion por pantalla 
    de datos en la tabla tesis */  
	if (mysql_query(con, "SELECT * FROM tesis")){
    	finish_with_error(con);
  	} 
	MYSQL_RES *result = mysql_store_result(con);
	if (result == NULL){
		finish_with_error(con);
	}
	int num_fields = mysql_num_fields(result);
    MYSQL_ROW row;
    while ((row = mysql_fetch_row(result))){ 
	for(int i = 0; i < num_fields; i++){ 
		printf("%s ", row[i] ? row[i] : "NULL"); 
		} 
	}
	 mysql_free_result(result);
	 mysql_close(con);// Termino de uso de Socks//
\end{lstlisting}
\label{anexc:1}

%\lstinputlisting[caption="Módulo de transición desde LoRa a IPv6"]{}
%\lstinputlisting[language=C, firstline=512, lastline=600]{/home/w01f/Documents/respaldo-raspberry/home/pi/lora/project_ipv6/src/project_ipv6.c}