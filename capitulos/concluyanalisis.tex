%%%%CONCLUSION%%%%%%
\begin{justify}
\chapter{Conclusiones}
En cuanto al diseño e implementación del modelo de simulación desarrollados en este proyecto, se puede decir que la implementación del modelo de simulación, logra imitar de forma análoga el comportamiento de los dispositivos LoRa en relación a su funcionamiento lógico y físico, dado que logra simular cada fase de comunicación de los dispositivos LoRa, como la fase de emparejamiento, transmisión/retransmisión y la fase de descanso en nodos. De esta misma forma se imitó el funcionamiento físico de los dispositivos, lo que se pudo realizar implementando funciones que simulan de forma análoga el comportamiento de la tasa adaptativa de envío de datos. En las funciones que simulan el comportamiento del \gls{adr}, se analiza la distancia entre el nodo y el \textit{gateway} de forma dinámica para asignar el canal (\gls{sf}) adecuado para dicha , cambiando con esto la tasa de envío de datos correspondiente al canal asignado. Adicionalmente se logró implementar una función que simula la pérdida de paquetes, donde en base a un \gls{per} asignado, esta función calcula si es necesario o no descartar un paquete en cada canal para cumplir con el \gls{per} asignado.\\
La implementación de este modelo de simulación muestra el cómo se realiza la comunicación de los dispositivos LoRa en una red de topología estrella, donde es posible probar tanto distribuciones de nodos en los distintos canales o \gls{sf}, como también el rendimiento de los dispositivos al agregarle variables reales como el porcentaje de la pérdida de paquetes, o la asignación dinámica de la tasa de envío, tomando la distancia que posee el nodo hasta el \textit{gateway}.\\
En la implementación del módulo de transición LoRaWAN/IPv6, se logró la correcta adquisición de la información, tanto de los dispositivos de la red LoRa, como del \textit{Payload} que se busca transmitir hacia un servicio de carácter web como lo es una Base de datos. En este módulo de transición, es posible agregar, dependiendo de las necesidades del desarrollador, una capa de cifrado a los datos, usando OpenSSL u TLS para comunicarse con los servidores objetivo, donde en el caso de agregar cifrado, se estaría además añadiendo una capa de seguridad adicional a la existente actualmente (cifrado sobre el mensaje), al usar el método de retransmisión de paquetes hacia el computador remoto, el cual sólo realiza una condición \textit{Pass Through} en el \textit{firewall} local.\\
\noindent
En relación al desarrollo del modelo de simulación, este se caracteriza por aportar la capacidad de realizar pruebas de conectividad de dispositivos LoRa, con distintas distribuciones y con diferentes configuraciones de parámetros (\gls{per}, tasa de envío de datos, etc.) en un ambiente virtual, obteniendo resultados análogos a los que se obtendrían con una medición en un ambiente real. No obstante, la topología abarcada en este proyecto es la topología estrella, este modelo no admite distribuciones estrella-de-estrellas ni variantes más complejas, por lo que este aporte iría enfocado en redes simples con un \textit{gateway} único en la red.\\
En cuanto al módulo de transición, este otorga la capacidad de poder interconectar de forma directa una red LoRa con un servicio de red, teniendo la posibilidad de agregar o no, una capa de seguridad extra al canal de transmisión hacia Internet.
\end{justify}