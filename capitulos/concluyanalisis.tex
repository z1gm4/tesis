%%%%CONCLUSION%%%%%%
\begin{justify}
\chapter{Conclusiones}
En cuanto a el diseño e implementación del modelo de simulación desarrollados en este proyecto, se puede decir que la implementación del modelo de simulación, logra imitar de forma análoga el comportamiento de los dispositivos LoRa en relación a su funcionamiento lógico y físico, dado que logra simular cada fase de comunicación de los dispositivos LoRa, como la fase de emparejamiento, transmisión/retransmisión y la fase de descanso en nodos. De esta misma forma se logró imitar el funcionamiento físico de los dispositivos, logrando implementar funciones que logran imitar de forma análoga el comportamiento de la tasa adaptativa de envío de datos, donde se analiza la distancia entre el nodo y el Gateway de forma dinámica para asignar el canal (Spreading Factor) adecuado para dicha distancia cambiando con esto la tasa de envío de datos correspondiente al canal asignado. Adicionalmente se logró implementar una función simula la pérdida de paquetes, donde en base a un PER asignado, esta función calcula si es necesario o no descartar un paquete en cada canal para cumplir con el PER asignado.\\
La implementación de este modelo de simulación muestra el como se realiza la comunicación de los dispositivos LoRa en una red de topología estrella, donde es posible probar tanto distribuciones de nodos en los distintos canales o Spreading Factor, como también el rendimiento de los dispositivos al agregarle variables reales como el porcentaje de la pérdida de paquetes o la asignación dinámica de la tasa de envío tomando la distancia que posee el nodo hasta el Gateway.\\
En la implementación del módulo de transición LoRaWAN/IPv6, se logró la correcta adquisición de la información, tanto de los dispositivos de la red LoRa, como del Payload que se busca transmitir, hacia un servicio de carácter WEB como lo es una Base de datos. Dado que este módulo de transición es implementado en base al uso de sockets, es posible agregar una capa de cifrado a los datos usando OpenSSL u TLS para comunicarse con los servidores objetivo, agregando una capa de seguridad más elevada que la existente actualmente al usar el método de retransmisión de paquetes desde el computador remoto.\\
\noindent
En relación al desarrollo tanto del modelo de simulación,este se caracterizan por aportar la capacidad de poder realizar pruebas de conectividad de dispositivos LoRa, con distintas distribuciones y con diferentes configuraciones de parámetros (PER, tasa de envío de datos, etc.) en un ambiente virtual, obteniendo resultados análogos a los que se obtendrían con una medición en un ambiente real. No obstante, la topología abarcada en este proyecto es la topología estrella, este modelo no admite topologías estrella-estrella ni variantes más complejas, por lo que este aporte iría enfocado en redes simples con un Gateway único en la red.\\
En cuanto al módulo de transición, este otorga la capacidad de poder interconectar de forma directa una red LoRa con un servicio de red, pero teniendo la posibilidad de agregar una capa de seguridad extra al canal, agregando TLS o SSL dependiendo de las necesidades del usuario, y de la compatibilidad del aplicativo al cual se desea conectar. Sin embargo, para el desarrollo de este módulo, no fue contemplada la adquisición de la información contenida en la cabecera del mensaje que envía el nodo al Gateway, por lo que no fue posible realizar una correlación nodo-mensaje en este módulo para luego poder mapear la red LoRa desde un servicio WEB.
\end{justify}