\begin{justify}
\chapter[Búsqueda Bibliográfica]{Búsqueda Bibliográfica}
\label{ch:estadodelarte}

\section{Estado del Arte}
En los últimos años, se ha generado un gran interés por el estudio sobre las tecnologías LPWAN, dado que son capaces de transmitir datos de forma inalámbrica a grandes distancias (las que en algunos casos alcanzan los $15km$), pero a una baja tasa de transmisión de datos. Este tipo de tecnología inalámbrica provee un sinfín de aplicaciones posibles, y dada su gran versatilidad y autonomía puede ser usado desde el área de la medicina, como hasta en el sector agrario para control de cosechas. Estas características junto a la capacidad de transmisión y autonomía de estos dispositivos, ha llamado la atención de muchos investigadores, quienes se han dedicado a demostrar tanto las capacidades de los LoRa, como los límites de estas. Entregando así bastante información sobre su comportamiento físico y lógico, junto a la comprobación del cumplimiento y/o superación de las especificaciones técnicas publicadas por los fabricantes. Adicionalmente, se han realizado investigaciones sobre la integración del Internet de las cosas o ``\textit{IoT}'' en los dispositivos LoRa con el fin de realizar una mayor integración de soluciones inteligentes que pueden desarrollarse sobre la base de los LoRaWAN. Bajo este contexto, ha nacido la necesidad de un ambiente virtual para la utilización de estos dispositivos, donde algunos investigadores han realizado avances en el modelado de redes ALOHA bajo la utilización de software de simulación de eventos discretos, con el fin de poder reducir los costos asociados a la investigación junto a facilitar un medio portable  para el estudio de estos dispositivos.\\
Los estudios que abarcan estos avances aquí expuestos se dividen en los siguientes temas:
\begin{itemize}
\item Funcionamiento de los dispositivos LoRA.\\
\item Limitaciones en el uso de LoRaWAN.\\
\item Modelado de redes ALOHA.\\
\item Integración de IoT en LoRaWAN.\\
\item Métodos de programación en C orientado a redes.\\
\end{itemize} 

\subsection{Funcionamiento de los dispositivos LoRa}
Los dispositivos LoRa son dispositivos diseñados para la comunicación inalámbrica a largas distancias basados en el antiguo protocolo ALOHA. Esta tecnología permite conectar dispositivos a una distancia de $15km$ en zonas sub urbanas, y hasta $2km$ en zonas urbanas según \cite{Sornin} y \cite{Sornin2}. Además estos poseen una autonomía mucho mayor dado que utilizan un método optimizado de asignación de ventanas de tiempo para la transmisión, basado en su predecesor ALOHA en su variante Slotted \cite{NORMAN}, el que se caracteriza por generar un sistema de calendarización desde el Gateway a sus nodos, donde se asignan ventanas de transmisión por fracciones de tiempo, minimizando de esta forma las colisiones de paquetes entre nodos. Los nodos, en caso de no poseer activas ventanas de transmisión, estos se colocan en estado de hibernación o reposo, hasta que llega el segundo de transmitir datos al Gateway. Adicionalmente se expone en \cite{modulation}, que LoRaWAN implementa un manejo multi-canales en base a espectros de señal, los que poseen una diferencia tal en su frecuencia, que no poseen interferencia entre ellos, cada una de estas diferenciales de frecuencia usadas del espectro expandido son denominadas Spreading Factor, donde es asignado un identificador a cada banda de frecuencia disponible por cada canal de comunicación. Esta técnica llamada Spreading Spectrum permite que las comunicaciones de LoRaWAN posean una mayor resistencia frente a interferencias de medios externos, el operar con una baja densidad espectral de energía, como también el otorgar un canal seguro de comunicación, no permitiendo el acceso a oyentes no autorizados. Asimismo cada canal posee un determinado sacrificio o ganancia de la velocidad de transmisión por distancia de transmisión, esto es posible mediante el uso de métodos de tasa adaptativa de transmisión (ADR). El Gateway LoRa asigna un Spreading Factor o una lista de canales disponibles en caso de ser posible, para que el nodo pueda comunicarse con el Gateway.\\
Las especificaciones aquí expuestas, son utilizadas para este proyecto de título, para poder comprender el comportamiento, tanto lógico, como físico de los dispositivos LoRaWAN, con el fin último de poder modelar estos comportamientos con una herramienta de simulación de redes y lograr una simulación análoga al comportamiento de los dispositivos LoRa en el ámbito físico y lógico.

\subsection{Limitaciones en el uso de LoRaWAN}
Los dispositivos LoRa poseen dentro de sus especificaciones técnicas expuestas en \cite{orange}, la definición de parámetros que definen a la larga el comportamiento de estos artefactos, como por ejemplo la distancia máxima de transmisión, las bandas de frecuencia en las que trabaja, etc. Estas capacidades han sido puestas a prueba por \cite{Xavier}, para conocer los límites de las capacidades de LoRaWAN. Dentro de los ámbitos puestos a prueba en \cite{Xavier}, está la limitación de la capacidad del canal y el máximo tamaño de una red LoRa (sobre la base del número de nodos y Gateways en la red). Dentro de los aportes de estos investigadores, está el determinar la diferencia real entre las tasas de transmisión, entre los diferentes Spreading Factors, tomando en cuenta el tiempo que toma el paquete en viajar en el aire hacia su destino ``\textit{Time on Air}'', como también el tamaño del payload, esta diferencias pueden apreciarse en la Fig~\ref{arte:1}, donde al usar un Spreading Factor que posee menor alcance de transmisión efectiva, es posible notar una transmisión más expedita del payload enviado, aún así, a tamaños mayores de payload, y en el caso contrario, en los Spreading Factors de mayor alcance de transmisión, el tiempo de viaje a destino es mucho mayor que en el resto de Spreading Factors de menor alcance. Adicionalmente, los investigadores indican que el desempeño de una red LoRa, depende de la fracción de tiempo en que el canal está ocupado (``\textit{Duty Cycle}''), y de las colisiones inherentes de un enlace basado en el protocolo ALOHA, como puede verse en la Fig~\ref{arte:2} al aumentar el número de nodos LoRa (representado con N en el gráfico), disminuye considerablemente la cantidad de paquetes recibidos, esto en parte se debe a la saturación del canal producida por colisiones de paquetes enviados al Gateway, junto con las retransmisiones provenientes de los nodos, que terminan por amplificar el efecto de las colisiones iniciales produciendo una saturación del canal. Esta saturación del canal ocurre por cómo está diseñada la transmisión y retransmisión de paquetes del protocolo ALOHA\cite{NORMAN} donde si un paquete es enviado, y en un tiempo determinado el nodo no recibe un acuse de recibo del paquete enviado, este transmitirá en la siguiente ventana disponible de tiempo el mismo paquete hasta que reciba una respuesta, el problema es que al aumentar el número de nodos las ventanas de tiempo disminuyen tanto en número como en duración de estas, por lo que es más frecuente las colisiones dentro de estas.\\
\begin{figure}[!ht]
\centering
\includegraphics[scale=0.4]{images/estadoarte1.png}
\caption{Gráfica de Tamaño de Payload MAC por Tiempo que toma el paquete en llegar a destino (``\textit{Time on Air}''). Fuente:\cite{Xavier}}
\label{arte:1}
\end{figure}
\begin{figure}[!ht]
\centering
\includegraphics[scale=0.4]{images/estadoarte2.png}
\caption{Gráfica de Paquetes generados en una hora por número de paquetes recibidos en una hora. Fuente: \cite{Xavier}}
\label{arte:2}
\end{figure}
Por otra parte, en la investigación de \cite{Juha}, se colocaron a prueba las capacidades de transmisión, pero en relación a la distancia máxima de transmisión, en zonas llanas sin interferencia por edificios o personas, y en zonas urbanas, con el objetivo de determinar si los LoRa cumplen con las especificaciones que entregan, y por otra parte, averiguar si pueden sobrepasar estas especificaciones técnicas entregadas por los fabricantes y descubrir nuevos límites para el uso de los dispositivos LoRa.\\
En las pruebas realizadas en \cite{Juha}, el Gateway es situado a $24m$ sobre la altura del mar con una antena bi-cónica de $2dbi$ de ganancia, donde los nodos con los que se comunicará, uno estará navegando sobre un bote en un ambiente libre de edificios y árboles, mientras que el otro estará sobre el techo de un automóvil, donde cada nodo se alejará cada vez más sobre su medio de transporte, para evaluar si es posible transmitir efectivamente dentro de los parámetros que indican los fabricantes de LoRa, o si incluso es posible superar estas especificaciones. Los resultados obtenidos de las pruebas de medición de la investigación \cite{Juha}, pueden verse en las Fig~\ref{arte:3} y Fig~\ref{arte:4}\\
\begin{figure}[!ht]
\centering
\includegraphics[scale=0.6]{images/estadoarte3}
\caption{Cantidad de paquetes transmitidos, recibidos y Packet Loss por rango de distancia en medición hecha por nodo sobre auto. Fuente:\cite{Juha}}
\label{arte:3}
\end{figure}
\begin{figure}[!ht]
\centering
\includegraphics[scale=0.6]{images/estadoarte4}
\caption{Cantidad de paquetes transmitidos, recibidos y Packet Loss por rango de distancia en medición hecha por nodo sobre bote. Fuente:\cite{Juha}}
\label{arte:4}
\end{figure}
Los resultados obtenidos por \cite{Juha}, son de gran ayuda para el desarrollo de este proyecto, dado que entrega una guía de valores esperados al momento de realizar mediciones con los dispositivos reales. Asimismo, otorga parámetros como una tasa de errores en paquetes (PER) para integrar al simulador con el fin de acercarlo más al funcionamiento real de los dispositivos. Y de la misma manera \cite{Xavier}, entrega los conocimientos suficientes para poder modelar el comportamiento real de la saturación del canal en base a las retransmisiones de los nodos, y las colisiones generadas.
\subsection{Modelado de redes ALOHA}

El estándar LoRaWAN trabaja sobre la base del protocolo ALOHA, el cual es uno de los primeros protocolos de comunicación orientados a la conexión inalámbrica dispositivos. En \cite{NORMAN} explican que ALOHA trabaja sobre un canal de radio frecuencia, que utiliza tanto para el envío, como para la recepción de datos, por lo que para aumentar la transmisión efectiva de datos (``\textit{Throughput}'') el protocolo ALOHA puro transmite datos en ventanas de tiempo aleatorias, y en caso de que el receptor no responda, en más del doble del tiempo que debiera responder con un acuse de recibo (ACK), el dispositivo emisor retransmitirá este paquete, y procederá con ese procedimiento hasta que reciba un acuse de recibo, con lo que concluye su transmisión de datos. Por otra parte Slotted ALOHA  o ``Ranurado'' implementa discretas ventanas de tiempo, donde el sólo podrá enviar o recibir datos, al inicio de una ventana de tiempo, con lo que se minimiza el número de colisiones.\\
Slotted-ALOHA al ser un protocolo que permite las comunicaciones inalámbricas, con un buen manejo de colisiones, gracias a sus ventanas de tiempo discretas, investigadores como \cite{Abdullah}, se han interesado en generar ambientes virtuales para simular el comportamiento de estos dispositivos, con el fin de poder realizar pruebas y estudios sobre el uso de este protocolo, sin necesidad de tener que instalar una red ALOHA, con todos los costos tanto de tiempo como de dinero que conllevan.\\
En relación a los aportes de \cite{Abdullah}, este investigador realiza tres modelos que permiten el múltiple acceso de computadores mediante Slotted ALOHA, con el fin de otorgar modelos de simulación del protocolo para que estudiantes puedan comprender de mejor forma el como se comunican los dispositivos ALOHA. Estos modelos de simulación imitan el comportamiento lógico del protocolo, entregando una herramienta útil para la comprensión del funcionamiento de redes de este tipo. Bajo este contexto, de la misma forma que nació la necesidad de modelar el funcionamiento del protocolo ALOHA, investigadores como \cite{simulato} y \cite{simubook} realizaron avances en el modelado de redes de sensores a gran escala, lo que entrega modelos que describen no sólo el funcionamiento del protocolo en condiciones ideales, si no que también la capacidad de modelar condiciones de borde y situaciones diferentes de las ideales, lo que lo asemeja más al modelo al comportamiento real de los dispositivos.\\
Por otra parte, \cite{Murat} entrega un estudio sobre los diferentes simuladores de redes que existen, sus ventajas y desventajas frente al resto y los protocolos que acepta cada una de estas herramientas computacionales y sobre las capacidades de integración entre protocolos y tecnologías que ofrecen los distintos Frameworks de simulación.\\
En relación con este proyecto, todas las investigaciones mencionadas en este apartado, entregan la base teórica para el entendimiento del protocolo ALOHA, el cual es la base del funcionamiento de los dispositivos LoRa, y junto a esto las investigaciones de \cite{Abdullah} y \cite{simulato} entregan técnicas y métodos de como modelar tanto funcionamiento ideal y comportamiento lógico de protocolos en base a máquinas de estado ( en el caso de una simulación de eventos discretos), como también sobre el modelado de redes con parámetros como PER, Atenuación de señal, entre otros elementos que permiten el acercar un modelo de simulación, al comportamiento análogo de dispositivos reales. Finalmente el artículo de \cite{Murat}, aporta con información sobre los diferentes frameworks de simulación de redes, lo que permite tener una cantidad aceptable de información para poder decidir que software utilizar a la hora de desarrollar el simulador de dispositivos LoRa.

\subsection{Integración de IoT en LoRaWAN}

%%tesis tomas, header compression y aplicaciones iot de lora%%
Los dispositivos LoRa, se comunican en base a transmisión de datos por radio frecuencia de un salto, esto junto con sistemas de modulación LoRa. En el escenario de desear conectar los datos obtenidos por los nodos LoRa con una aplicación WEB, sistema no podría enviar los datos a través de internet dado que no posee la capacidad de enrutamiento, no obstante el Gateway LoRa es capaz de realizar una retransmisión de los paquetes recibidos hacia un backend (computador remoto) y con esto permitir la conectividad con aplicaciones WEB para la recepción de datos, aunque hasta el momento no es posible saber que nodo mandó un dato específico, o enviar datos desde el Gateway de forma directa hacia una dirección IP, esta carencia de los dispositivos LoRa limita las posibles aplicaciones con estos dispositivos. Bajo este contexto, algunos investigadores \cite{lowpan} han realizado modelos de compresión de cabeceras del protocolo IPv6 para dispositivos bajo el estándar LPWAN, donde se propone un modelo llamado 6LoWPAN que elimina ciertos campos de la cabecera del protocolo IPv6, de acuerdo a la configuración de los primeros $2bytes$, con el fin de reducir el tamaño de la cabecera del paquete para direcciones IPv6 unicast, donde puede comprimirlos a $64 bits$ o $16 bits$ dependiendo de que campos se omiten, u omitir los campos opcionales por completo. En \cite{tomas}, este modelo teórico es aplicado a una red LoRa donde la cabecera IPv6 comprimida junto con el payload deseado, se enviará a través del Payload de dispositivos LPWAN hacia el Gateway. Una vez llegado al Gateway, en este se construirá un paquete IPv6 con las especificaciones entregadas en la cabecera comprimida, donde adicionalmente se agregará la información relacionada al payload del nodo emisor, y sus direcciones de red, las cuales al ser pertenecientes a una red LoRa, serán ingresadas en una tabla de enrutamiento que poseerá una equivalencia entre direcciones IPv6 y dirección de red LoRa, con el fin de poder luego enviar datos directamente desde los nodos hacia direcciones IPv6 y viceversa entregando la nueva función al Gateway LoRa de enrutador de mensajes.\newpage
\noindent
Cabe decir de que el Gateway LoRa para esta funcionalidad, necesita estar conectado al computador remoto que le provee de conexión a Internet, por medio de una interfaz virtual TUN/Puente, la que dejará pasar directamente los datos desde la red LoRa desde la interfaz TUN virtual (donde está conectado el Gateway LoRa), hacia la interfaz con salida a Internet del computador remoto. Si bien, la retransmisión de paquetes que posee nativa, realiza el mismo procedimiento, con la diferencia que ahora el Gateway LoRa creará el paquete de datos que se enviara, y este sólo será retransmitido hacia Internet, en cambio antes, el computador remoto tenía la labor de crear dicho paquete de datos, lo que entrega mayor autonomía e independencia de uso a los dispositivos LoRa.\\
Estos avances sobre la compresión de la cabecera IPv6 para dispositivos LPWAN, y la integración de IPv6 mediante el uso de estos modelos de compresión en una red LoRa, entregan a este proyecto el conocimiento necesario para poder diseñar un módulo que permita actuar de gestor de transición de una comunicación directa desde el Gateway hacia servicios locales dentro del computador remoto (sean bases de datos o aplicaciones WEB locales), y un intermediario mas seguro para la retransmisión de los paquetes, dado que al comunicarse directamente con la API dentro del módulo bajo una conexión por socket asegura una conexión segura por medio del uso de protocolos como TLS que dan la capacidad de cifrar la información y así evitar que un tercero atente contra la privacidad, integridad o disponibilidad de los datos.
\newpage
\subsection{Métodos de programación en C orientado a redes.}
%%libro de sockets, integracion de modulo%%
En muchas aplicaciones WEB, se han de utilizar scripts en sus servicios WEB para que realicen acciones deseadas por los desarrolladores, como también muchas veces son desarrollados scripts en el backend, para realizar procedimientos como entrega de datos, creación de objetos, etc.El problema nace en que estos scripts son desarrollados en lenguajes como Python, Java o C, los que no tienen una orientación a red, si no, más bien a objetos, lo que dificulta el traspaso de datos obtenidos por los scripts hacia las aplicaciones WEB o sus servicios relacionados.\\
Bajo este contexto, nace la necesidad de que los scripts contenidos en aplicaciones WEB, se conecten tanto a API's como a otros servicios existentes. Para poder llevar a cabo esta tarea, es necesario el uso de ``sockets'', los que constan de la apertura de conexiones bidireccionales a puertos determinados, que permiten la interacción en lectura como escritura contra servicios y aplicaciones WEB (Bases de datos, API's, archivos de configuración, registros de depuración, etc). En \cite{network} enseñan múltiples técnicas, métodos y plantillas de código que permiten establecer conexión con servicios de red mediante el uso de sockets en el lenguaje C.\\
Los conocimientos aportados por este libro ayudaron en el desarrollo del módulo de transición de LoRaWAN a IPv6 en este proyecto, aportando métodos y funciones que otorgan la capacidad de conectar el Gateway LoRa con servicios y aplicaciones WEB. Esta conexión es generada en el mismo script que maneja la recepción de los mensajes de los nodos LoRa con el protocolo IPv6 integrado, resultando de esta forma, un sistema íntegro y con un nivel mayor de seguridad para poder conectar los datos obtenidos en la red LoRa con una aplicación WEB y todos sus servicios de red asociados.
\end{justify}