%%glosario%%%
%definicion de terminos
%%glosario
\newglossaryentry{iotg}{name={Internet of Things}, description={IoT es el término usado en ciencias de la computación, para referirse a la integración de los protocolos IPv4 o IPv6, en dispositivos que no fueron inicialmente diseñados para realizar una conectividad mediante el uso de Internet, esto se realiza con el fin de poder ampliar la diversidad de usos y aplicaciones que pueden llegar a tener dispositivos de esta naturaleza}}

\newglossaryentry{lpwang}{name={LPWAN}, description={Las LPWAN, son todas aquellas redes inalámbricas, diseñadas para permitir comunicaciones de largo alcance a una baja tasa de envío de bits}}

\newglossaryentry{lorawang}{name={LoRaWAN}, description={LoRaWAN, es el protocolo utilizado en las redes LPWAN de la familia de los LoRa, el que tiene la cualidad de optimizar el funcionamiento de ALOHAnet utilizando una programación de mensajes en base al tiempo, y junto con esto, el uso de multi canales de comunicación para evitar colisiones y asegurar una conectividad con el menor grado de pérdidas. Aunque, también este acrónimo es utilizado para hacer referencia a las redes LPWAN de la familia de los LoRa}}

\newglossaryentry{socket}{name={Socket}, plural={Sockets}, description={Los Sockets, son interfaces de programación, para el uso de los dos tipos de protocolos involucrados en la capa de transporte. Estos protocolos son TCP y UDP.}}

\newglossaryentry{perg}{name={PER}, description={El PER es el porcentaje de paquetes que llegaron a destino con error en el transcurso de una transmisión de datos. Los motivos de la recepción con errores del paquete puede deberse a interferencias de señal, bits corruptos, entre otros motivos.}}

\newglossaryentry{adrg}{name={ADR}, description={El ADR es una técnica de adaptación de la tasa de envío de datos, a fin de aumentar o disminuir la distancia máxima de transmisión y con esto mejorar la calidad del enlace.}}

\newglossaryentry{sfg}{name={SF}, description={Los \textit{Spreading Factors}, son índices para la denominación dada a los canales de bandas de frecuencia de espectro expandido, utilizados por los dispositivos LoRa. Estos canales poseen frecuencias base específicas, como también una tasa de envío de datos específica para cada canal, e inclusive una distancia máxima de transmisión asociado a la potencia del enlace.}}

\newglossaryentry{tung}{name={TUN}, description={TUN, es el nombre para los dispositivos virtuales de red, los que son operan completamente en base a \textit{Softwares}, a diferencia de los dispositivos de red que poseen un respaldo por \textit{Hardware}. Estos dispositivos, son comúnmente usados como interfaces puente entre dos o más redes, con el fin de otorgar conectividad entre redes, sin la necesidad de utilizar equipamiento físico.}}

%%acrónimo
\newglossaryentry{tun}{type=\acronymtype, name={TUN}, description={\textit{network TUNnel}},  first={TUN(\textit{network TUNnel})\glsadd{tung}}, see=[Glosario:]{tung}}


\newglossaryentry{sf}{type=\acronymtype, name={SF}, description={\textit{Spreading Factor}},  first={SF(\textit{Spreading Factor})\glsadd{sfg}}, see=[Glosario:]{sfg}}

\newglossaryentry{adr}{type=\acronymtype, name={ADR}, description={\textit{Adaptative Data Rate}},  first={ADR(\textit{Adaptative Data Rate})\glsadd{adrg}}, see=[Glosario:]{adrg}}

\newglossaryentry{per}{type=\acronymtype, name={PER}, description={\textit{Package Error Rate}},  first={PER(\textit{Package Error Rate})\glsadd{perg}}, see=[Glosario:]{perg}}

\newglossaryentry{iot}{type=\acronymtype,  name={IoT}, description={\textit{Internet of Things}}, first={IoT(\textit{Internet of Things})\glsadd{iotg}}, see=[Glosario:]{iotg}}

\newglossaryentry{lorawan}{type=\acronymtype,  name={LoRaWAN}, description={\textit{Long Range Wide-Area Network}}, first={LoRaWAN(\textit{Long Range Wide-Area Network})\glsadd{lorawang}}, see=[Glosario:]{lorawang}}

\newglossaryentry{lpwan}{type=\acronymtype,  name={LPWAN}, description={\textit{Low Power Wide-Area Network}}, first={LPWAN(\textit{Low Power Wide-Area Network})\glsadd{lpwang}},  see=[Glosario:]{lpwang}}
